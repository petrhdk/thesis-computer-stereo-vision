Die Projektive Geometrie beschäftigt sich als Teilgebiet der Geometrie mit der projektiven Transformation, die eine dreidimensionale Szene auf eine zweidimensionale Ebene abbildet. Die hier geltenden mathematischen Zusammenhänge lassen sich unter der Annahme des Lochkamera-Modells (vgl. \ref{subsec:kameramodell} (\emph{Kameramodell})) auf die Bildentstehung in einer Kamera anwenden.\kleinerabstand

\noindent Unter Einsatz der linearen Algebra und mit dem Zusammenspiel von euklidischen Koordinaten und homogenen Koordinaten (vgl. \ref{subsec:homogene_koordinaten} (\emph{Homogene Koordinaten})) lässt sich dieser Prozess der Bildentstehung anschaulich mit mathematischen Gleichungen beschreiben, welche in \ref{subsec:bezugssysteme_bildentstehung} (\emph{Bezugssysteme und Bildentstehung}) hergeleitet werden.\kleinerabstand

\noindent Im Unterabschnitt \ref{subsec:kamerakalibrierung} (\emph{Kamerakalibrierung}) wird dann auf das Verfahren eingegangen, mit dem die unbekannten Parameter des Bildentstehungsprozesses ermittelt werden.\kleinerabstand

\noindent Schließlich wird in \ref{subsec:triangulation} (\emph{Triangulation}) mithilfe der erlangten Grundlagen und Gleichungen der Bildentstehungsprozess umgekehrt betrachtet. Aus einem Bildpunkt wird also der zugehörige Lichtstrahl rekonstruirt. Durch Triangulation mit den zwei verschiedenen Lichtstrahlen der beiden Kameras kann letztendlich auch die Position des gesehenen Objektes bestimmt werden.\kleinerabstand

\noindent Grundlage für alle Betrachtungen dieses Abschnitts stellen \cite{cyrill_stachniss_hc, cyrill_stachniss_cei, cyrill_stachniss_dlt} dar.