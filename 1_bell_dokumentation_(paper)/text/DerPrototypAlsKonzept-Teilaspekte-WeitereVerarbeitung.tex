Mit den erhaltenen Positionen der Verkehrsteilnehmer in potenziell gefährlicher Situation kann letztendlich je nach Anwendungsfall beliebig verfahren werden.\kleinerabstand

\noindent Für den Haupteinsatzzweck, der Sendung von Warnnachrichten an sich in der Nähe befindende Kraftfahrzeuge, muss das Programm an dieser Stelle eine Risikobewertung durchführen und eine Schnittstelle für die Kommunikation zu den genannten Kraftfahrzeugen implementiert haben. Der Endpunkt der Datenverarbeitung ist dann beim Informationsempfänger. In Autos können die Positionsinformationen zum Beispiel auf internen Navigationskarten und in modernen Head-up-Displays visualisiert werden.\\
Ausführlichere Betrachtungen dieser Art sollen allerdings, wegen der schon in \ref{sec:beschreibung_gesamtsystem} skizzierten Gründe nicht Teil dieser Forschungsarbeit sein. Das Prototypenkonzept betrachtet nur den Weg zum Erhalten der dreidimensionalen Ortung beliebiger Verkehrsteilnehmer.
