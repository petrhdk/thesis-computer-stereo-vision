Bei der sogenannten \emph{Triangulation} wird ein Obejekt durch die Kenntnis seiner Sensorkoordinaten zweier Kameras räumlich geortet (engl.: \en{Stereo Vision}). Denn, wie allgemein bekannt, kann eine einzelne Kamera nur Richtungsinformationen und keine Tiefeninformationen über einen gesehenen Gegenstand liefern. Durch zwei in ihrer Position verschiedene Kameras wird die Tiefeninformation aber berechenbar.\kleinerabstand

Als erstes muss der Bildentstehungsprozess umgekehrt werden. Man erhält die Richtungs-Halbgerade, auf der der räumliche Punkt aus Sicht einer der Kameras liegen muss.\\
Dazu wird der Bildpunkt in Sensorkoordinaten $\mathrm{x}=(x,y,1)^T$ zunächst mit der Inversen der Kalibrierungsmatrix $K$ multipliziert. Man gelangt formal vom Sensor-Koordinatensystem zum Kamerakoordinatensystem.
\begin{equation}
	\boldsymbol{K}^{-1}\cdot {^s\xH}
\end{equation}
Wichtig ist hier, dass man aufgrund der $3{\times}3$-Dimension von $K$ nicht die vier homogenen Koordinaten für die dreidimensionale Position des Punktes erhält\footnote{dafür müsste $K^{-1}$ an dieser Stelle eine $4{\times}3\;$-Matrix sein}, sondern nur die ersten drei dieser eigentlich vier homogenen Koordinaten. Ohne die vierte Koordinate kann man die ersten drei nicht zur einfachen Form normalisieren\footnote{vgl. Gleichung \ref{eq:einfache_form}} und bleibt mit dem bloßen Verhältnis von $X$, $Y$ und $Z$ des Punktes aus Sicht der Kamera zurück.
\begin{equation*}
	\boldsymbol{K}^{-1}\cdot{^s\xH} {\:\:}={\:\:} \lambda\: {^k\XC} \tag{\theequation$^\prime$}
\end{equation*}
$\lambda\: {^k\mathsf{X}}$ entspricht nun der Richtung, in der der Punkt aus Sicht der Kamera liegt, bzw. der angesprochenen Halbgeraden.\\
Um den Richtungsvektor aus dem Kamera-Koordinatensystem nun noch in das Welt-Koordinatensystem zu übertragen, muss nur die Inverse $R^T$ der entsprechenden Drehmatrix $R$ linksseitig anmultipliziert werden.
\begin{equation}\label{eq:gleichung_richtungsvektor}
	\boldsymbol{a} = \boldsymbol{R}^T \boldsymbol{K}^{-1} \xH
\end{equation}
$a$ stellt jetzt im Welt-Koordinatensystem dar, in welche Richtung vom Kameraprojektionszentrum aus der gesehene Punkt liegt.
Die Koordinaten Punktes im Raum $\mathsf{F}$ würden sich dann mithilfe der Position des Kameraprojektionszentrums $\mathsf{X}_0$ so angeben lassen:
\begin{equation}\label{eq:gleichung_halbgerade}
	\boldsymbol{\mathsf{F}} = \XC_0 + \lambda\:\boldsymbol{a}
\end{equation}\kleinerabstand

Jetzt werden beide Kameras zusammen betrachtet und es wird berücksichtigt, dass sich die Halbgeraden der beiden Kameras wegen Messungenauigkeiten bei der Kalibrierung und endlich genauer Erkennung durch den \en{Object Detection Algorithmus} nicht genau schneiden. Es ist daher der Punkt X gesucht, der zu beiden Halbgeraden den kürzesten und gleichen Abstand hat. Er lässt sich als Mittelpunkt der Abstandsstrecke zwischen beiden Halbgeraden konstruieren. Die Abstandstrecke schneidet die die Halbgeraden in den Punkten F und G und steht auf beiden senkrecht.\mittelgrosserabstand

\begin{figure}[H]
	\centering
	\def\svgwidth{10cm}
	\import{image/triangulation/}{triangulation.pdf_tex}
	\caption{Geometrie der Triangulation}
	\label{fig:triangulation}
\end{figure}

Die beschriebenen Zusammenhänge werden im Folgenden durch Gleichungen ausgedrückt und schließlich zur Lösung gebracht. Parameter der ersten Kamera werden mit einem Strich und die der zweiten mit zwei Strichen notiert.\kleinerabstand

{\noindent}Die Richtungsvektoren $a$, $b$ der beiden Halbgeraden im Welt-Ko\-or\-di\-na\-ten\-sys\-tem ergeben sich analog zu \eqref{eq:gleichung_richtungsvektor} als:
\begin{equation}
	\boldsymbol{a} = \boldsymbol{R}'^T \boldsymbol{K}'^{-1} \xH'
	\qquad\quad
	\boldsymbol{b} = \boldsymbol{R}''^T \boldsymbol{K}''^{-1} \xH''
\end{equation}
F und G liegen jeweils auf einer der beiden Halbgeraden und ihre Koordinaten ergeben sich analog zu \eqref{eq:gleichung_halbgerade} daher als:
\begin{equation}
	\boldsymbol{\mathsf{F}} = \XC'_0 + \lambda\:\boldsymbol{a}
	\qquad\quad
	\boldsymbol{\mathsf{G}} = \XC''_0 + \mu\:\boldsymbol{b}
\end{equation}
Die Orthogonalität der Abstandstrecke zu den beiden Halbgeraden lässt sich jeweils über das Skalarprodukt beider Richtungsvektoren ausdrücken, welches nach Definition $0$ sein muss\footnote{bei einem Schnittwinkel von $90^\circ$ zweier Vektoren $u$ und $v$ wird deren Skalarprodukt $u \cdot v \:=\: |u||v|\cos(90^{\circ}) \:=\: 0$}.\kleinerabstand
%\begin{center}\setlength{\fboxsep}{0.2cm}\fbox{\parbox{.92\columnwidth}{
%\begin{equation}
%	\text{Skalarprodukt bei $\theta{=}90^{\circ}$:}\qquad
%	u \cdot v {\:}={\:} |u||v|\cos(90^{\circ}) \:=\: 0
%\end{equation}
%}}\end{center}

\noindent Es lässt sich demnach folgendes Gleichungssystem aufstellen, dessen Lösung die unbekannten Variablen $\lambda$ und $\mu$ sind.
\begin{align*}
	&\left|
	\begin{array}{lcl}
		(\boldsymbol{\mathsf{F}}-\boldsymbol{\mathsf{G}})\cdot\boldsymbol{a} & = & 0\\
		(\boldsymbol{\mathsf{F}}-\boldsymbol{\mathsf{G}})\cdot\boldsymbol{b} & = & 0
	\end{array}		
	\right| \\[0.2cm]
%
	&\left|
	\begin{array}{lcl}
		(\XC'_0+\lambda\:\boldsymbol{a} - \XC''_0-\mu\:\boldsymbol{b}) \cdot\boldsymbol{a} & = & 0\\
		(\XC'_0+\lambda\:\boldsymbol{a} - \XC''_0-\mu\:\boldsymbol{b}) \cdot\boldsymbol{b} & = & 0
	\end{array}
	\right| \\[0.2cm]
%
	&\left|
	\begin{array}{lllcl}
		(\XC'_0 -  \XC''_0)\cdot\boldsymbol{a} & + \lambda\:\boldsymbol{a}^2 & - \mu\:\boldsymbol{a}\cdot\boldsymbol{b} & = & 0\\
		(\XC'_0 -  \XC''_0)\cdot\boldsymbol{b} & + \lambda\:\boldsymbol{a}\cdot\boldsymbol{b} & - \mu\:\boldsymbol{b}^2 & = & 0
	\end{array} 
	\right| \\[0.2cm]
%
	&\left|
		\begin{array}{rrlcl}
			\boldsymbol{a}^2\cdot\lambda & -\boldsymbol{a}\cdot\boldsymbol{b}\cdot\mu & +(\XC'_0 -  \XC''_0)\cdot\boldsymbol{a} & = & 0\\
			\boldsymbol{a}\cdot\boldsymbol{b}\cdot\lambda & -\boldsymbol{b}^2\cdot\mu & +(\XC'_0 -  \XC''_0)\cdot\boldsymbol{b} & = & 0
		\end{array} 
		\right|
\end{align*}\vspace{0.15cm}
\begin{equation}
	\begin{pmatrix}
		\boldsymbol{a}^2 & -\boldsymbol{a}\cdot\boldsymbol{b}\\
		\boldsymbol{a}\cdot\boldsymbol{b} & -\boldsymbol{b}^2
	\end{pmatrix}
	\:\begin{pmatrix}\lambda\\\mu\end{pmatrix}
	\:\:=\:\:
	\begin{pmatrix}
		(\XC''_0 -  \XC'_0)\cdot\boldsymbol{a}\\
		(\XC''_0 -  \XC'_0)\cdot\boldsymbol{b}
	\end{pmatrix}
\end{equation}\vspace{0.2cm}

{\noindent}Die gesuchten euklidischen Koordinaten des Punktes im Raum $\mathsf{X}$ ergeben sich dann, wie erläutert, als arithmetisches Mittel von $\mathsf{F}$ und $\mathsf{G}$.
\begin{equation}
	\boldsymbol{\mathsf{X}} \:\:=\:\:  \sfrac{1}{\mspace{2mu}2} ( \XC'_0{+}\lambda\boldsymbol{a} \:+\: \XC''_0{+}\mu\boldsymbol{b} )
\end{equation}
