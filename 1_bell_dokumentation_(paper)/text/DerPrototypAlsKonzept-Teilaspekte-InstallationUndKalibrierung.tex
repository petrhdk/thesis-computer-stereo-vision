Bei der Positionierung der Kameras ist Einiges zu beachten.\kleinerabstand

\noindent Zu aller erst sollten die Kameras hauptsächlich den gleichen Sichtbereich abdecken, da die nicht-motorisierten Verkehrsteilnehmer nur geortet werden können, wenn sie in beiden Kamerabildern sichtbar sind.\kleinerabstand

\noindent Auf keinen Fall dürfen beide Kameras sehr dicht nebeneinander positioniert sein. Dadurch würden große Ungenauigkeiten in der Ortung entstehen, weil bei der Triangulation als Methode der \en{Computer Stereo Vision} der Schnittpunkt der beiden jeweils zu den Bildpunkten gehörenden Kamerastrahlen berechnet wird\footnote{vgl. \ref{subsec:triangulation} (\emph{Triangulation})}. Wenn die Strahlen aufgrund dieser schlechten Kamerapositionierung annähernd parallel wären, würden kleine zufällige Fehler zu einer enormen Abweichung von der realen Position des zu ortenden Objektes führen.\kleinerabstand

\noindent Die Entfernung von der zu beobachtenden Verkehrssituation muss abhängig von dem einzusehendem Bereich gewählt werden, ist aber nahezu beliebig, wenn für besondere Fälle einfach spezielle Kameratypen eingesetzt werden. So bieten sich für sehr kurze Distanzen Weitwinkelobjektive und für weiter entfernte Kamerastandorte Teleobjektive an.\kleinerabstand

\noindent Für den Standort der Kameras sollte stets eine erhöhte Lage präferiert werden, weil daraus eine bessere Übersicht über die Verkehrssituation resultiert. Seltener werden Verkehrsteilnehmer von anderen verdeckt, oder Ähnliches. Diese Eigenschaft ist aber ebenso essentiell für das \en{Correspondence Matching}. Die Kameras dürfen keinesfalls auf Höhe der Verkehrsteilnehmer angebracht werden, weil die Komplanaritätsbedingung sonst immer erfüllt ist und damit kein Kriterium für die Entscheidung einer Übereinstimmung zwischen zwei Detektionen mehr darstellt.\kleinerabstand

\noindent Wie in \ref{subsec:kameramodell} (\emph{Kameramodell}) erläutert, muss weiterhin auf die Vermeidung von stark Bild-verzerrenden Kameraobjektiven geachtet werden, da das verwendete Lochkameramodell sonst nicht mehr anwendbar wäre. Diese Eigenschaft lässt sich mit der Beobachtung von in der Welt geraden Linien auf dem Bild der Kamera feststellen. Wenn diese geraden Linien auch auf dem Bild der Kamera gerade Linien bleiben\footnote{en.: \en{straight line preserving mapping}}, dann lässt sich das Lochkameramodell anwenden. Wenn diese allerdings als Kurven abgebildet werden, liegt eine Verzerrung vor.\mittelgrosserabstand
%Bildentstehungprozesse mit nicht-linearen Fehler stellen per se kein Problem dar, erfordern aber eine weiterführende, tiefgreifende Kalibrierung, die im Rahmen dieser Arbeit nicht betrachtet wird.


Wenn final installiert und justiert, dürfen die Kameras bestmöglich nicht mehr verstellt werden, weil nur solange keine Erneuerung der vorgenommenen Kalibrierung notwendig ist, wie Position, Orientierung und innere Parameter der Kamera identisch bleiben.\\
Für die Kalibrierung werden die Grundlagen aus \ref{subsec:kamerakalibrierung} (\emph{Kamerakalibrierung}) aufgegriffen. Es werden mindestens sechs, besser mehr, Kalibrierungspunkte benötigt, für die händisch die dreidimensionale Position im lokalen Welt-Bezugssystem und die Pixel-Position auf jeweils einem Testbild bestimmt wird. Kritische Konfigurationen der Kalibrierungspunkte im Raum, wie zum Beispiel die Beziehung, in einer Ebene zu liegen, müssen entsprechend der Hinweise vermieden werden.\\
Um die gerade erwähnten und für die Kalibrierung notwendigen dreidimensionalen Positionen der Kalibrierungspunkte im Raum zu erhalten, sind grundsätzlich verschiedene Wege denkbar. Eine in der Anwendung einfache und deshalb für diesen Prototypen gewählte Methode wurde in den theoretischen Grundlagen in \ref{sec:kugelkoordinaten_und_ihre_transformationen} (\emph{Kugelkoordinaten und ihre Transformationen}) vorgestellt. Hierbei werden ledeglich die geographischen Koordinaten und Höhen über dem Nullniveau der Kalibrierungspunkte gemessen und daraus ein lokales Koordinatensystem konstruiert.\\
Gleichzeitig könnten auch andere manuelle Verfahren zur Vermessung der Kalibrierungspunkte angewandt werden, die sich der Wissenschaft der Geodäsie zuordnen lassen. Hierbei kommen in der Regel spezielle Laser-Messgeräte, sogenannte \emph{Tachymeter}, zum Einsatz, um die räumliche Szene mit den Kalibrierungspunkten vor Ort auszumessen. Ein Vorteil davon wäre, dass sich nicht auf Kalibierungspunkte beschränkt werden muss, die auf den Satellitenbildern sichtbar sind, und eventuell eine ungünstige Konstellation aufweisen. Als Kalibierungspunkte könnten auch beliebig platzierbare Objekte, wie z.B. ein Tachymeter-Reflektorstab, verwendet werden.\\
Beide Kameras durchlaufen nun einzeln den hergeleiteten Kalibrierungsalgorithmus aus \ref{subsec:kamerakalibrierung} (\emph{Kamerakalibrierung}) und man erhält die äußeren und inneren Parameter jener.
