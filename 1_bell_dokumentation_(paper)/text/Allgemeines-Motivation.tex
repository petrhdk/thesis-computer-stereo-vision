Ein großer Forschungsbereich bei der Anwendung von Computertechnologien im Alltag ist der Verkehr. Fast jeder hat schon von der Idee der selbst fahrenden Autos gehört und zahlreiche Automobilhersteller und Wissenschaftler konnten bisher erstaunliche Ergebnisse vorstellen. Assistenzsysteme werden heute schon in fast allen neuen Autos verbaut.\kleinerabstand

\noindent Doch viel weniger Aufmerksamkeit als das automatisierte Fahren \cite{unterschied_fahren} bekommt das Gebiet des vernetzten Fahrens \cite{unterschied_fahren}. Hier wird berücksichtigt, dass Straßenverkehr viel effizienter stattfinden kann, wenn Kraftfahrzeuge nicht nur auf sich bedacht (autonom \cite{unterschied_fahren}) fahren, sondern mit anderen Verkehrsteilnehmern und der Verkehrsinfrastruktur (z.B. Ampeln, digitale Verkehrstafeln, Parkautomaten oder Verkehrskameras) kommunizieren (\en{Car2Car Communication} und \en{Car\-2\-In\-fra\-struc\-ture Communication} \cite{c2c_c2i_heise, c2c_c2x_koblenz}). Durch den Informationsaustausch wird ein sowohl sichererer als auch flüssigerer Verkehr ermöglicht und vorrangig gilt: geteilte Informationen sind besser als verschlossene Informationen. Denn je mehr Informationen ein Verkehrsteilnehmer oder z.B. eine Ampelanlage über die aktuelle Verkehrslage zur Verfügung hat, desto qualifizierter kann eine Entscheidung getroffen werden \cite{c2c_c2i_heise, c2c_c2x_koblenz}.\kleinerabstand

\noindent Wegen seinem großen Potenzial wächst das Interesse am vernetzten Fahren in der Forschung und in der Wirtschaft stetig. Beispielhaft zu nennen ist hier das Forschungs-Projekt \emph{Ko-HAF}\footnote{``kooperatives, hochautomatisiertes Fahren''}, bei dem vier deutsche Univeritäten mit zahlreichen Industriepartnern unter staatlicher Förderung eine führende Kompetenz im Bereich automatisiertes Fahren und vernetztes Fahren entwickeln \cite{kohaf_projektpartner, kohaf_ergebnisse}.\kleinerabstand

Auch diese Arbeit soll dabei helfen, die Rolle des vernetzten Fahrens, besonders der \en{Car2Infrastructure Communication}, im Verkehr der Zukunft zu erörtern.\\
Besonders wichtig ist weiterhin, dass die Ergebnisse der Forschung in erster Linie den nicht-motorisierten Verkehrsteilnehmern zu Gute kommen, da sich der Großteil der Verkehrsforschung um Verbesserungen für den Automobilverkehr dreht.
