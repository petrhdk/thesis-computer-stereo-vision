

Nachdem im ersten großen Teil der Dokumentation\footnote{Kapitel \ref{chap:theoretische_grundlagen} (\emph{Theoretische Grundlagen})} mit der detaillierten Betrachtung theoretischer Grundlagen wichtige Funktionsalgorithmen und technische Hilfsmittel erarbeitet wurden, konnte aus diesen im zweiten Hauptteil\footnote{Kapitel \ref{chap:der_prototyp_als_konzept} (\emph{Der Prototyp als Konzept})} ein allgemein einsetzbares Prototypenkonzept zur automatisierten optische Erfassung und dreidimensionale Ortung von nicht-motorisierten Verkehrsteilnehmern mithilfe von zwei Verkehrskameras zusammengesetzt werden, welches sich schließlich im dritten großen Abschnitt\footnote{Kapitel \ref{chap:der_prototyp_in_der_praxis_versuch} (\emph{Der Prototyp in der Praxis, Versuch})} in Python implementiert und in einem konkreten Versuch getestet als funktionstüchtiges und vielversprechendes System erwies.\kleinerabstand

Der erarbeitete Prototyp stellt sich damit als wertvoller Beitrag zur Forschung am vernetzten Fahren, zum einen als Demonstrator und zum anderen als Grundlage für real einsetzbare Systeme, heraus.\\
Auf der Basis dieser Forschungsarbeit könnte aus dem Prototypen, aus der Idee, eine hocheffiziente Lösung zur Erhöhung der Verkehrssicherheit im Rahmen der \en{Car2Infrastructure Communication} reifen. Dazu wären Verbesserungen an der grundsätzlichen Funktion Prototyps in verschiedenen Aspekten interesssant beziehungsweise notwendig zu erörtern:
\begin{itemize}
	\item die Vereinfachung des Kalibrierungsalgorithmus, sodass zum Beispiel weniger Kalibrierungspunkte benötigt werden oder die inneren Kameraparameter vollautomatisch unter Nutzung von Abgleichsalgorithmen wie dem \emph{RANSAC}-Algorithmus \cite{ransac} erhalten werden können
	\item die Erhöhung der Zuverlässigkeit der Objekterkennung; hier wäre unter Anderem der Einsatz von mehreren Kameras gleichzeitig denkbar
	\item die Einbeziehung weiterer Informationen zum Abgleichen der Detektionen, wie die Seitenverhältnisse der Detektionsrechtecke oder die Bildinformationen der Detektionen selber\footnote{zwei Bilder können mit sogenannten \en{Content-based image retrieval}-Algorithmen (\en{CBIR}) \cite{cbir}, die üblicherweise für Bilder-Suchmaschinen eingesetzt werden, auf Übereinstimmung geprüft werden}
	%\item die Effizienzsteigerung des Programmcodes oder anderweitige Verkürzung der Laufzeit, zum Beispiel durch speziell angepasste Computer-Hardware
	\item die Weiterverarbeitung der Analysedaten, zum Beispiel zur Ermittlung von Geschwindigkeitsinformationen oder zum Voraussage des Standorts mithilfe der bisherig bekannten Bewegungsrichtung
\end{itemize}
