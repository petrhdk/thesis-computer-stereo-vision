Das Konzept sieht, entsprechend der allgemeinen Abbildung \ref{fig:stereo_vision} der Zielstellung, ein System aus zwei fest installierten Verkehrskameras an einer bestimmten Verkehrssituation gepaart mit einem Computer vor, auf dem ein im Weiteren einfach ``Programm'' genannter Programmcode für die gewünschte Funktionalität sorgt.\kleinerabstand

\noindent Damit das Aussenden von Warnnachrichten an z.B. Autos grundsätzlich möglich ist, muss eine Verarbeitung der Kameradaten in Echtzeit erfolgen. Hierfür wird grundlegend zwischen zwei Ansätzen unterschieden. Zum einen kann die Rechenlogik in Form einer kleinen Computer-Lösung direkt mit den Kameras vor Ort eingesetzt werden. Es ist aber auch der Einsatz von \en{Cloud Computing} denkbar, bei dem die Kameradaten zum Auswerten von leistungsstarken Rechennetzwerken über das Internet vermittelt werden. Dies scheint besonders in Hinsicht auf den neuen Mobilfunkstandard 5G attraktiv, der besonders niedrige Latenzen und hohen Datendurchsatz bei der Datenübertragung verspricht \cite{5g}.\kleinerabstand

\noindent Für das System an sich ist letztendlich auch noch wichtig, wie die Kommunikation zu den Informations-Empfängern stattfindet. Dies soll jedoch im Sinne der aktuellen Forschung nicht ausführlich konzeptioniert oder entwickelt werden, da an dieser Stelle schon ein anderes Forschungsgebiet beginnt. 

%Es lässt sich aber absehen, dass die Übermittlung der Warnnachrichten im Falle der lokalen Computer-Lösung drahtlos direkt zum Rezipienten, z.B. über das Mobilfunknetz, und beim Einsatz von \en{Cloud Computing}