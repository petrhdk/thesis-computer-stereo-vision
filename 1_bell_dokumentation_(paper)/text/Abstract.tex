\vfill

{\bfseries\LARGE Abstract}\\[25pt] 

\begin{small}
Besides automated driving, particularly studies on connected cars are a major trend in today's traffic research. The focus of this principle lies on the exchange of information between the vehicles as well as between the vehicles and the road infrastructure. This is because a broader information basis will enable all parties involved to make more qualified decisions, which will have a positive impact on traffic flow and safety.\kleinerabstand

\noindent The present scholarly work aimed to develop a prototype for the automated optical detection and three-dimensional localization of non-motorized vehicles using two traffic cameras. Subsequently, a test in a real traffic situation was prepared, carried out and evaluated, for which the program code was to be implemented in the Python programming language.\kleinerabstand

\noindent To detect the non-motorized vehicles, an open source object detection algorithm based on deep learning is continuously applied to the images of the two traffic cameras. With the determined pixel position of an interesting object of the scene on the pictures of these two cameras, and by incorporating all features of the cameras obtained by calibration, the exact position of the road user in the three-dimensional space is concluded. Within the framework of Car2Infrastructure Communication, this information could then be sent as an alert message to motorized vehicles in the immediate vicinity and processed further there.\kleinerabstand

\noindent The key aspect of the research was the elaboration of the necessary mathematical foundations for the described three-dimensional localization process of an object based on two overlapping camera images. However, many other theoretical and practical aspects played a role in the development of the prototype concept and the conduct of the experiment. Among other things, this included camera calibration, arranging the found interesting objects into corresponding pairs between the camera images, operating with spherical coordinates, the use of external cartography and visualization software, as well as programming in Python.\kleinerabstand

\noindent The developed prototype proved oneself to be a valuable contribution to research on connected traffic, on the one hand as a demonstrator in the sense of the experiment and on the other hand as a basis for systems that can be used in real applications. 
Further research could focus on simplifying the calibration process, increasing the accuracy and reliability of 3D positioning, gathering speed information and predicting road users' positions.\kleinerabstand
\end{small}

\vfill
