Grundsätzliches Ziel des Versuches soll es sein, die Funktionstüchtigkeit des Prototyps im Sinne eines \en{Proof of Concept} an einem konkreten Einsatz zu belegen. Die Leistungsbewertung und -optimierung des Systems sei nur zweitrangig von Bedeutung. Ursächlich hierfür ist, dass bisher kein direkt vergleichbares Erkennungssystem bekannt ist und es daher in erste Linie um die Potenzbeurteilung des Konzepts geht.\kleinerabstand

\noindent Demnach reicht es für den Versuch, den Programmcode zur Verarbeitung der Kameradaten im Nachhinein auszuführen, und nicht wie bei einem realen Anwendungsfall in Echtzeit\footnote{vgl. \ref{sec:beschreibung_gesamtsystem} (\emph{Beschreibung des Gesamtsystems})}. Dadurch wird viel Aufwand bei der Versuchsdurchführung gespart, und die Funktionstüchtigkeit des Systems kann nichtsdestotrotz gleichartig beurteilt werden.\kleinerabstand

In den einzelnen Unterabschnitten dieses Kapitels wird folgend auf die für den Versuch zu erfüllenden Aufgaben eingegangen.
