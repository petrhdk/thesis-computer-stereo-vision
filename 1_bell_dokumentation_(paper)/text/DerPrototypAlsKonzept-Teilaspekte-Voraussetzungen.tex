Um den Prototyp nun detaillierter zu entwickeln sind zunächst die nötigen Voraussetzungen zu betrachten.\kleinerabstand

\noindent In erster Linie wird eine geeignete Verkehrssituation benötigt. Diese sollte sinngebend für den Einsatzes des Systems besondere Gefahrenstellen für nicht-motorisierte Verkehrsteilnehmer aufweisen. Das heißt über Bereiche verfügen, die nicht aus jeder Verkehrslage einsehbar sind. In der Regel handelt es sich dabei um Kreuzungen mit eng parkenden Autos.\\
Günstig wäre auch, dass die Verkehrsinfrastruktur Installationsmöglichkeiten für die Kameras und eventuell den Verarbeitungscomputer, wie zum Beispiel eine Ampelanlage, zu Verfügung stellen kann. Auf die Installation der Kameras wird weiter im nächsten Abschnitt eingegangen.\kleinerabstand

\noindent Neben den offensichtlichen Voraussetzungen, wie die rechtliche Absicherung für die automatisierte, anonyme Verkehrsüberwachung und dem Vorhandensein eines Kommunikationsweges zu den vorgesehenen Empfängern der Warninformationen, stellt der Prototyp keine weiteren einschränkenden Bedingungen für den Einsatz. Er ist grundsätzlich sehr allgemein und vielseitig einsetzbar. Sogar die Installation von mehreren solchen Systemen an einer einzigen Kreuzung ist denkbar.

%\noindent Die mit Abstand wichtigste Voraussetzung stellt zweifellos das Vorliegen der Hardware, in Form der beiden Kameras und eventuell des lokalen Computers, und der Software-Implementierung entsprechend der gegeben Rechnerarchitektur und Programmschnittstellen dar. Diese muss vor
