Im laufenden Betrieb wendet das Programm nun fortlaufend jeweils für jedes neue Paar von Kamerabildern ein Erkennungsalgorithmus\footnote{en.: \en{Object Detection} Algorithmus} an. Die Auswahl an diesen ist sehr groß, die Entscheidung sollte von verschiedenen Faktoren abhängig gemacht werden.\\
Es muss zum einen die geforderte Zuverlässigkeit und Präzision gegen die dafür notwendige Rechenleistung abgewogen werden. Weiterhin gibt es sehr viele verschiedene Grundtypen von Erkennungsalgorithmen, die jeweils einsatzspezifische Vor- und Nachteile haben. Entscheidend ist auch die Frage, ob auf einen vortrainierten\footnote{beim \en{Deep Learning} oder allgemein \en{Machine Learning} wird die Funktionalität des Erkennens von Objekten durch Training auf bzw. mit bestimmten Datensätzen erlernt} Algorithmus zurückgegriffen werden soll, oder ob nicht ein eigenes Training mit angepassten Datensätzen, vielleicht sogar eine ganz eigene Implementation eines \en{Object Detection} Algorithmus, für höhere Präzision und Zuverlässigkeit zu präferieren ist. Das muss gegen den potenziell sehr hohen Arbeitsaufwand abgewogen werden.\\
Es ist schließlich auch noch zu betonen, dass in diesem Gebiet der Forschung sehr schnell Fortschritte gemacht werden und deshalb kein Erkennungsalgorithmus für immer der beste Kandidat sein wird.\kleinerabstand




