Im Abschnitt \ref{sec:projektive_geometrie} (\emph{Projektive Geometrie}) und \ref{sec:epipolargeometrie} (\emph{Epipolargeometrie}) findet jeweils ein Welt-Bezugssystem Verwendung, um Positionen und Richtungen im Raum, z.B. mit euklidischen Koordinaten, anzugeben. Es wird im Folgenden lokales Bezugssystem genannt.\\
Es stellt sich die Frage nach der Erlangung bzw. Konstruktion dieses lokalen Bezugssystems, wodurch wird es definiert.
Hier bietet sich die Verwendung von geographischen Koordinaten\footnote{das heißt die Positionsangabe im globalen Bezugssystem mit Längengrad, Breitengrad und der Höhe über dem Nullniveau} an, weil diese sowohl eine einfache Vermessung der Kalibrierungspunkte (eben mit geographischen Koordinaten) ermöglichen, als auch praktisch für die Ausgabe der letztendlich ermittelten Positionen der Verkehrsteilnehmer zu verwenden sind.\kleinerabstand

\noindent Dafür ist nun die Hin- und Rück-Transformation zwischen den geographischen Koordinaten im globalen Bezugssystem und den entsprechenden euklidischen Koordinaten im lokalen Bezugssystem gesucht. Im Folgenden werden diese in einzelnen Teilschritten hergeleitet.\kleinerabstand
