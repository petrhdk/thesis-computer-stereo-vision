Für die Implementation des für den Versuch benötigten Programmcodes wurde die Programmiersprache Python gewählt. Diese zeichnet sich durch eine einfache Syntax und eine breite Unterstützung von Programmbibliotheken aus, was ein effizientes Programmieren unterstützt und im Allgemeinen besser lesbaren Code produziert.\\
Zur einfachen Handhabung von Vektoren und Matrizen sowie deren Operationen und Zerlegungen wurde die Programmbibliothek NumPy \cite{numpy} verwendet, eine der meist verwendeten Python Erweiterungen für wissenschaftliches Rechnen \cite{numpy}. Dadurch wird auf eine zuverlässige und im Vergleich zur eigenen Implementierung effiziente und zeitsparende Lösung zurückgegriffen.\kleinerabstand

\noindent Ein weiterer Aspekt der Software-Implementation ist, dass der gesamte Programmcode in englischer Sprache verfasst wurde, um neben dem besseren Zusammenspiel mit der Programmiersprache hauptsächlich die Zugänglichkeit des Codes für die internationale Gemeinschaft zu ermöglichen. Außerdem verbessert sich die Lesbarkeit allgemein.\kleinerabstand

\noindent Als Erkennungsalgorithmus\footnote{wie in \ref{subsec:erkennung} (\emph{Erkennung}) beschrieben} wurde eine quelloffene und vortrainierte Implementierung \cite{mobilenetssd_github} des \en{Single Shot Detector} Algorithmus \cite{ssd} in Form eines \en{Mobile Net} \cite{mobilenet} eingesetzt. Diese hält ein für die Anwendung in dieser Forschung perfektes Mittelmaß zwischen Effizienz und Präzision beziehungsweise Zuverlässigkeit. Bei der Programmierung zum Einsatz des genannten Erkennungsalgorithmus' galt \cite{pyimagesearch} als Vorlage.\kleinerabstand

\noindent Als weiteres Hilfsmittel der Programmierung wurde auf die verbreitete Ver\-si\-o\-nie\-rungs-Software Git \cite{git} zurückgegriffen, welche weitläufigen Einsatz in fast allen heutigen Software-Entwicklungen aufweist. Sie eignet sich zum dezentralen, verteilten und nicht-linearen Entwickeln und stellt unter Nutzung eines \en{Online Remote Repositories}\footnote{ ein \en{Git Repository} ist, vereinfacht gesagt, ein Depot für alle Projektdateien;\\ein \en{Remote Repository} ist eine dynamische Kopie dieses Depots bei einem Cloud-Anbieter} einen von überall und jedem zugänglichen Speicherort dar.\kleinerabstand

\noindent Der gesamte beim Versuch eingesetzte Programmcode, inklusive Annotationen, findet sich im Anhang der Dokumentation wieder.
