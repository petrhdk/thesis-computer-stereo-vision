Die Epipolargeometrie betrachtet den Aufbau zweier Kameras, die eine bestimmte, gleiche Szene beobachten, und trifft Aussagen über die geometrischen Beziehungen zwischen ihren überlappenden Bildern.\\
Für diese Arbeit soll sie beim Abgleichen der Detektionen (en.: \en{Correspondence Matching}) zum Einsatz kommen, das heißt zum Ordnen der in den beiden Kamerabildern jeweils erkannten Verkehrsteilnehmer zu korrespondierenden Paaren, wofür eine Möglichkeit der Überprüfung für die Zusammengehörigkeit zweier Objekt-Entdeckungen zwischen den Kamerabildern benötigt wird\footnote{vgl. \ref{subsec:abgleichen_der_entdeckungen} (\emph{Abgleichen der Entdeckungen})}.\\
Dafür lässt sich die sogenannte Komplanaritätsbedingung nutzen, welche die zentrale Gesetzmäßigkeit der Epipolargeometrie darstellt.\kleinerabstand

\begin{figure}[H]
	\centering
	\def\svgwidth{12cm}
	\import{image/epipolargeometrie/}{epipolargeometrie.pdf_tex}\kleinerabstand
	\caption{Übersicht der Epipolargeometrie}
	\label{fig:epipolargeometrie}
\end{figure}
